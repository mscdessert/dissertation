\documentclass[12pt,a4paper]{report}

\usepackage{url}
\usepackage{epcc}
\usepackage{graphics}

% This example file shows how a thesis can be laid out using Latex. It
% does not use any special local features so should be portable to other
% places.
% 
% When producing draft copies of a thesis you may want to print only
% selected pages of the thesis. To do this use the command
% 
% dvips -f -p 4 -n 3 myfile.dvi | lpr
% 
% where -p 4 means start printing at page 4 (ie the page that will be
% numbered 4, not necessarily the 4th page) and -n 3 means print 3 pages.
% This example will print pages 4, 5 and 6.
% 
% If you want to print the thesis and also save paper you can print more
% than one page on each sheet of paper. Use the command
% 
% dvips -f myfile.dvi | psnup -2 | lpr
% 
% to print 2 pages per sheet. psnup can take values 2, 4, 8, or 9.
%
% To produce a PDF version you can create a PostScript copy first
%
% dvips -f myfile.dvi > myfile.pdf
%
% and then convert it
%
% distill myfile.ps
%
% or you can go straight to PDF
%
% pdflatex myfile
%
% Note that pdflatex expects all included figures to be in PDF too. See
% the includegraphics command below.


% This document contains many cross-references and forward references,
% eg in constructing a table of contents, so Latex may need to be run
% twice to get all the references correct. If you need to run Latex twice
% you may get the warning:
% 
% LaTeX Warning: Label(s) may have changed. Rerun to get cross-refSerences right.


% the following 4 lines are the content of the smallmargins.sty file
% but including them explicitly makes this more portable.
%AC%\oddsidemargin=0.1in
%AC%\topmargin=-0.5in
%AC%\textheight=9in
%AC%\textwidth=6.25in

%AC%\parskip 10pt
%AC%\parindent 0in

\begin{document}


%AC%\pagestyle{myheadings}
%AC%\markright{D.~S.~Henty}

%\title{A Latex thesis example}
%\author{D.~S.~Henty}
%\date{\today}

%\maketitle

\pagenumbering{roman}

\title{Exploiting Private and Hybrid Clouds for Data-intensive Computing}
\author{Roshan.~T.~M}
\date{\today}

\makeEPCCtitle

\thispagestyle{empty}

\vspace{12cm}

\begin{center}

\large{MSc in High Performance Computing}

\large{The University of Edinburgh}

\large{Year of Presentation: 2012}

\end{center}

\newpage

\begin{abstract}

With the advent of the internet, social networks and
the ever increasing number of mobile devices which
help people stay networked, there has been an out burst of unstructured data.
[Amount of data produced] [Reference for unstructured]. There is a potential 
gold mine of information hidden within the data originating from social networks
like Twitter[], Facebook [], Google Plus[] etc. Social networks contain data 
about products people use, what they like and also about their expectations
about a product. A business management team 
can make valuable decisions if necessary information can be extracted from this data
and be made availale to them.
The challenge lies in extracting useful information within a reasonable 
amount of time. The key to extracting useful information with a reasonable
time-frame is distributed data processing.

This work investigates the usage of an open source cloud management toolkit,
OpenNebula [] , along with an open source distributed data processing framework,
Apache Hadoop [] to solve this problem. The benefits of using such an approach
along with the issues it raises are explained. OpenNebula is used to
create a private cloud which provides the infrastructure to build an Apache Hadoop cluster.
 
The distributed data processing framework is benchmarked for its performance
in the private cloud platform. The input/ouput throughput of the framework and
the performance of a word-count application obtained
with varying number of nodes in the cluster is measured and reported. 

Two applications that perform social media analytics are developed. The first
application extracts all the tweets[] related to London 2012 Olympics
from Twitter and measure the tone of the tweets which provides 
information on how people feel about the Olympics. The second application
extracts all the public tweets and then provides a summary of tweet source,
ie the device people use for tweeting. This information can be used by businesses
to identify potential marketing medium for a product or an event.

OpenNebula and Apache Hadoop proved to be one of pausible solutions to solving
data-intensive problems, however depending on the work-load of the application,
there are factors that limit performance because of Apache Hadoop cluster nodes having to share the physical
resources like hard-disk IO and the network bandwidth.


\end{abstract}

\pagenumbering{roman}

\tableofcontents
\listoftables
\listoffigures

\begin{titlepage}
\vspace*{2in}
% an acknowledgements section is completely optional but if you decide
% not to include it you should still include an empty {titlepage}
% environment as this initialises things like section and page numbering.
\section*{Acknowledgements}

This template is a slightly modified version of the one developed by
Prof. Charles Duncan for MSc students in the Dept. of Meteorology. His
acknowledgement follows:

{\em This template has been produced with help from many former students who
have shown different ways of doing things. Please make suggestions for
further improvements.}

\end{titlepage}

\pagenumbering{arabic}

\chapter{Introduction}

Information is wealth. This statement holds true in all aspects of
life. In the world of commerce, useful information about a product 
can help the management team make strategic business decisions
to improve the product and thus the customer experience. Security establishments
can take corrective measures or respond quickly to a possible security threat
if they have sufficient information before an intrusion. In both the cases
it is very important that the information is made available at the right time. 

\chapter{Background}

\section{Big-data}

{\bf Origin of the term Big-data}

With more than 2.5 quintillion bytes of data \cite{ibm:bigdata} being produced
every day, every now and then we run into a problem of not having 
sufficient computing resources to process and extract useful information
within the required time-frame.

The term big-data is time specific and broadly
big data can be defined at any point in time as ``data whose size forces us
 to look beyond the tried-and-true methods
that are prevalent at that time.`` In the early 1980s, it was a dataset that was
 so large that a robotic ``tape monkey'' was required to swap thousands of tapes in and out.
In the 1990s, perhaps, it was any data that transcended the bounds of Microsoft
Excel and a desktop PC, requiring serious software on Unix workstations to
analyze. Nowadays, it may mean data 
that is too large to be placed in a relational database and analyzed with the
help of a desktop statistics/visualization package-data, perhaps, whose
analysis requires massively parallel software running on tens, hundreds, or
even thousands of servers.\cite{jacobs2009pathologies}

The term big-data is more relevant today because 90\% \cite{ibm:bigdata}
of the data in the world today has been created in the last two years and 
hence big-data is getting a lot of hype from business publications and IT blogs
and the mainstream media. 

The shelf-life of the data is short so should be able to act on it before it expires.

Cost effectively process and analyze this database



Data sources

There is data being generated from everywhere: sensors used to gather geophysical information, 
posts to social media sites, pictures and videos posted on blogs, transaction
records from e-commerce sites, and mobile GPS signals to name a few.

According to IBM, the challenges of big-data spans three dimensions: Volume,
Velocity and Variety.\cite{ibm:bigdata}

Volume:

Enterprises are awash with ever-growing data of all types,
easily amassing terabytese-ven petabytes-of information.

Enterprises and even small businesses have access to data sources that
produce terabytes and petabytes of information. The volume of data 
gives rise to the challenge of processing the data before the shelf-life
of the data expires. 

Turn 12 terabytes of Tweets created each day into improved product sentiment analysis

Convert 350 billion annual meter readings to better predict power consumption


Velocity: Sometimes 2 minutes is too late. For time-sensitive processes such as catching fraud,
big data must be used as it streams into your enterprise in order to maximize its value.

Scrutinize 5 million trade events created each day to identify potential fraud

Analyze 500 million daily call detail records in real-time to predict customer churn faster


Variety: Big data is any type of data - structured and unstructured data such as text,
sensor data, audio, video, click streams, log files and more. New insights are found when
analyzing these data types together.

Monitor 100’s of live video feeds from surveillance cameras to target points of interest
Exploit the 80\% data growth in images, video and documents to improve customer satisfaction



%In any case, as analyses of ever-larger datasets become routine, the definition will continue to shift, but one
%thing will remain constant: success at the leading edge will be achieved by
%those developers who can look past the standard, off-the-shelf techniques
%and understand the true nature of the
%hardware resources and the full panoply of algorithms that are available to them



\cite{ibm:bigdata} 

{\bf Is Big-data refered only to unstructured data ?}


{\bf Statistics about the amount of data being generated from Social media}

Big-data is commonly refered to data-sets which cannot be processed on a
desktop workstation with in a resonable time-frame.

\section{Distributed data processing}

\section{Virtualization}

\section{Cloud Computing}

Cloud computing is a model where a pool of resources (hardware/software) are
managed to deliver them as virtualised resources on-demand.


Cloud Computing was a concept first publicly announced by John McCarthy in 1961 
(in a speech given to celebrate MIT's centennial). He saw a future in which computing 
power and even specific applications could be sold through the utility 
business model (like water or electricity) \cite{garfinkel1999architects} .
This concept and  supporting technologies evolved over the period and has taken us 
to the reality where computing resources are served as a utility. 
Technologies relating to CPU Power, network bandwidth, memory capacity 
and connectivity have improved dramatically in this period enabling Cloud Computing as 
a business model.

Cloud Computing is of substantial interest among IT and network technologists
and academics, as well as CIOs and business leaders, due to its potential to 
reduce cost and risk, increase revenue, and enhance total customer experience.\cite{weinman2011future}
It has been difficult to give a concise definition for Cloud Computing 
and related terms. According to the U.S. National Institute of Standards 
and Technology (NIST), \cite{mell2009nist}  "Cloud computing is a model for enabling convenient,
on-demand network access to a shared pool of configurable computing resources
(e.g., networks, servers, storage, applications, and services) that can 
be rapidly provisioned and released with minimal management 
effort or service provider interaction." 

The essential characteristics are \cite{mell2009nist} :

\begin{itemize}
  \item \emph{On-demand self services.} Provision resources without human
     interaction with service provider
  \item \emph{Broad network access.} Resources available over the network
  \item \emph{Resource pooling.} Resources are pooled to serve multiple 
        consumers using a multi-tenancy model
  \item \emph{Rapid elasticity.} Ability to increase or decrease resource consumption
  \item \emph{Measured service.} Resource usage can be monitored, controlled and reported.
\end{itemize}

\section{Apache Hadoop}

Apache Hadoop is a software framework for distributed processing of huge data-sets
using the MapReduce programming model. The Apache Hadoop 

\subsection{Apache MapReduce}

{\bf Google produce MapReduce to process data generated from their web-crawler}

\subsection{Apache HDFS}

\section{OpenNebula}

OpenNebula is an open source toolkit for building and managing virtualized
cloud infrastructure \cite{website:one}. The OpenNebula tookit provides a flexible
solution by providing interoperability with multiple hardware and software combinations.
It supports the Xen, KVM and VMWare hypervisors and provides both command line
and graphical interfaces to its features.


\section{Amazon Web Services (AWS)}

\subsection{Amazon EC2}

\chapter{Cloud Infrastructure Design}

\section{Cloud hardware and software stack}
\subsection{EPCC machines}
\subsection{Intel VT-X}
\subsection{Ubuntu 11.10}
\subsection{KVM}
\subsection{Libvirt}

\section{Cloud monitoring}
\subsection{OpenNebula Sunstone GUI}
\subsection{Ganglia}

\chapter{Hadoop Cluster Design}

\section{Hadoop software stack}
\subsection{Cloudera Hadoop distribution}
\section{Hadoop cluster on Private cloud}
\section{Hadoop cluster on Hybrid cloud}
\section{Running a MapReduce program}
\section{Benchmarking}


%\chapter{Experimental design}
\chapter{Twitter Data Analysis}

\section{Streaming API - Gathering the Tweets}

\subsection{Tweets included in Streaming API}

FROM: https://dev.twitter.com/docs/streaming-api/concepts
[
Only non-protected public accounts can create public statuses. 
Statuses, including replies and mentions, created by a public 
account are candidates for inclusion in the Streaming API. Statuses 
created by protected accounts and all direct messages are non-public 
and are currently not available via the Streaming API, but are 
available on User Streams and Site Streams.
]

\subsection{Twitter Policies}

May also want to include more details from

Twitter Search Rules and Restrictions; link same as below

FROM: http://support.twitter.com//forums/10713/entries/42646
[
In order to keep your search results relevant, Twitter filters search 
results for quality. Our search results will not include suspended accounts, 
or accounts that may jeopardize search quality. Material that degrades 
search relevancy or creates a bad search experience for 
people using Twitter may be permanently removed.
]
\subsection{Result Quality}

FROM: https://dev.twitter.com/docs/streaming-api/concepts
[
Both the Streaming API and the Search API filter, and on some end-points,
discard, statuses created by a small proportion of accounts based upon status
quality metrics. For example, frequent and repetitious status updates may,
in some instances, and in combination with other metrics, result in a
different status quality score for a given account. Results that are not selected
by user id, for example: samples and keyword track, are filtered by this
status quality metric. Results that are selected by user id, currently only
results from the follow predicate, are unfiltered and allow all
matching statuses to pass. If an expected user's statuses are not present in a
non-follow-predicate stream type, manually cross-check the user against Search
results. If the user's statuses are also not returned in Search, you can assume
that the user's statuses will not be returned by non-follow-predicated streams.
]

\section{Processing the Tweets}

\chapter{Results and Analysis}


\chapter{Conclusions}

This is the place to put your conclusions about your work. You can
split it into different sections if appropriate. You may want to include
a section of future work which could be carried out to continue your
research.

\appendix
% the appendix command just changes heading styles for appendices.

\chapter{OpenNebula Configuration Details}

Appendices should contain all the material which is considered too
detailed to be included in the main bod but which is, nevertheless,
important enough to be included in the thesis.

\chapter{Hadoop Cluster Configuration Details}

Some  people include in their \cite{fox2009above}thesis a lot of detail, particularly
computer code, which no-one will ever read. You should be careful that
anything like this you include \cite{website:aws} should contain some element
 of \cite{weinman2011future}uniqueness which justifies its inclusion.

\bibliographystyle{plain}
\bibliography{msc-biblio}
\end{document}

